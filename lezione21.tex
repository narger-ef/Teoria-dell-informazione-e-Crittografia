\section*{Lezione 21}
\addcontentsline{toc}{section}{Lezione 21}

\subsection*{Firme digitali e funzioni di hash}
\addcontentsline{toc}{subsection}{Firme digitali e funzioni di hash}

Una firma digitale è una quintupla $(P,A,K,S,V)$ dove:
\begin{itemize}
	\item $P$ è l'insieme di tutti i possibili messaggi che vorremmo firmare
	\item $A$ è l'insieme di tutte le possibili firme
	\item $K$ è l'insieme di tutte le possibili chiavi
	\item $S=\{\text{sig}:P\times K \rightarrow A\}$ è l'insieme delle funzioni di firma
	\item $V=\{\text{ver}:P\times A \times K \rightarrow \{T, F\}\}$ è l'insieme delle funzioni di verifica
\end{itemize}

Come nelle funzioni di cifratura e decifratura, scegliendo una chiave $k \in K$ essa diventa un parametro:
\begin{equation*}
	\text{sig}_k:P \rightarrow A
\end{equation*}
\begin{equation*}
	\text{ver}_k:P \times A \rightarrow \{T, F\}
\end{equation*}

La funzione di verifica mi dice sempre T o F.

La coppia $(x,y)$ è chiamata un messaggio firmato.

Diversamente da un autografo (che è fatto a mano), le firme:
\begin{itemize}
	\item la firma $y$ è separata dal documento $x$
	\item la firma non è sempre la stessa: dipende dal documento, quindi ce ne sono diverse per diversi documenti
\end{itemize}

La firma digitale serve per autenticare l'originatore del messaggio: solo chi conosce una certa informazione segreta può produrre la firma del messaggio.

Per firmare serve una chiave segreta, per verificare si usa una chiave pubblica; quindi il meccanismo è simile a quello dei crittosistemi a chiave pubblica ma al contrario.

Attenzione: autentica il generatore del messaggio ma non chi lo ha spedito!

\subsubsection*{Esempio}
Alice vuole firmare un messaggio $m$:
\begin{itemize}
	\item Lei scegli una coppia $(sig_k, ver_k)$ di algoritmi
	\item Mantiene segreta $sig_k$ e rende $ver_k$ pubblica
	\item Calcola $\sigma=sig_k(m)$
\end{itemize}
A questo punto Bob vuole verificare la firma prodotta da Alice:
\begin{itemize}
	\item Considera la coppia $(m, \sigma)$
	\item Prende l'algoritmo $ver_k$ e accetta la valida se e solo se $ver_k(m, \sigma) = T$.
\end{itemize}

min 22