\section*{Lezione 21}
\addcontentsline{toc}{section}{Lezione 21}

\subsection*{Firme digitali e funzioni di hash}
\addcontentsline{toc}{subsection}{Firme digitali e funzioni di hash}

Una firma digitale è una quintupla $(P,A,K,S,V)$ dove:
\begin{itemize}
	\item $P$ è l'insieme di tutti i possibili messaggi che vorremmo firmare
	\item $A$ è l'insieme di tutte le possibili firme
	\item $K$ è l'insieme di tutte le possibili chiavi
	\item $S=\{\text{sig}:P\times K \rightarrow A\}$ è l'insieme delle funzioni di firma
	\item $V=\{\text{ver}:P\times A \times K \rightarrow \{T, F\}\}$ è l'insieme delle funzioni di verifica
\end{itemize}

Come nelle funzioni di cifratura e decifratura, scegliendo una chiave $k \in K$ essa diventa un parametro:
\begin{equation*}
	\text{sig}_k:P \rightarrow A
\end{equation*}
\begin{equation*}
	\text{ver}_k:P \times A \rightarrow \{T, F\}
\end{equation*}

La funzione di verifica mi dice sempre T o F.

La coppia $(x,y)$ è chiamata un messaggio firmato.

Diversamente da un autografo (che è fatto a mano), le firme:
\begin{itemize}
	\item la firma $y$ è separata dal documento $x$
	\item la firma non è sempre la stessa: dipende dal documento, quindi ce ne sono diverse per diversi documenti
\end{itemize}

La firma digitale serve per autenticare l'originatore del messaggio: solo chi conosce una certa informazione segreta può produrre la firma del messaggio.

Per firmare serve una chiave segreta, per verificare si usa una chiave pubblica; quindi il meccanismo è simile a quello dei crittosistemi a chiave pubblica ma al contrario.

Attenzione: autentica il generatore del messaggio ma non chi lo ha spedito!

\subsubsection*{Esempio}
Alice vuole firmare un messaggio $m$:
\begin{itemize}
	\item Lei scegli una coppia $(sig_k, ver_k)$ di algoritmi
	\item Mantiene segreta $sig_k$ e rende $ver_k$ pubblica
	\item Calcola $\sigma=sig_k(m)$
\end{itemize}
A questo punto Bob vuole verificare la firma prodotta da Alice:
\begin{itemize}
	\item Considera la coppia $(m, \sigma)$
	\item Prende l'algoritmo $ver_k$ e accetta la valida se e solo se $ver_k(m, \sigma) = T$.
\end{itemize}

Eve vorrebbe:
\begin{itemize}
	\item Rompere totalmente il sistema: in qualche modo riesce a determinare la chiave segreta $k$ e quindi firmare documenti al suo posto
	\item Existential forgery: dopo aver osservato qualche coppia $(x_1, y_1), ..., (x_i, y_i)$ di messaggi con le rispettive firme, quando un nuovo messaggio arriva Eve può produrre una valida firma (valida solo per quel messaggio o per una classe di messaggi che hanno qualche similarità).
\end{itemize}

Quando Alice vuole mandare un messaggio firmato $m$ (non cifrato, solo firmato) a Bob, lei ha una funzione di cifratura $E_A$ e una funzione di decifratura $D_A$. Mantiene segreta $D_A$ mentre $E_A$ è pubblica.\\
A questo punto se vuole firmare $m$ usa $D_A(m)$ e la manda a Bob. Egli per verificare la firma prende l'algoritmo pubblico $E_A$ e controlla che \begin{equation*}
	E_A(D_A(m)) = m
\end{equation*}
E accetta la firma se e solo se la funzione di verifica ritorna vero.

Di solito però si firma un documento cifrato, quindi Alice calcola $D_A(E_B(m))$ utilizzando $E_B$ come algoritmo di cifratura pubblico di Bob. \\
Bob infine per trovare $m$ calcolerà:
\begin{equation*}
	D_B(E_A(D_A(E_B(m)))) = m
\end{equation*}
Quindi controlla la firma e poi decifra il messaggio.\\


Così però non va bene: se Eve intercettasse potrebbe prendere la firma e impersonificare Alice mandando $D_E(E_B(m))$. Quindi è meglio cifrare anche la firma, così solo Bob può leggere la firma.

\subsection*{Schema El Gamal}
\addcontentsline{toc}{subsection}{Schema El Gamal}

La base su cui è stato prodotto DSA (Digital Signature Algorithm) che è algoritmo usato ora.\\
Lo schema di El-Gamal non è deterministico: ogni messaggio (anche uguale) ha sempre una firma diversa. Come il suo crittosistema, la sicurezza si basa sui logaritmi discreti.
\begin{itemize}
	\item Si parte da un numero primo $p$
	\item Si trova un generatore $g$ del generatore del gruppo ciclico $\mathbb{Z}_p^*$
	\item I messaggi da cifrare sono elementi di $\mathbb{Z}_p^*$
	\item Le firme sono coppie $(\gamma, \delta)$ con $\gamma \in \mathbb{Z}_p^*$ e $\delta \in \mathbb{Z}_{p-1}$ dove $\mathbb{Z}_{p-1}$ non sarà un campo ($p-1$ è pari) ma un anello o boh 
\end{itemize}

Il problema, come nel crittosistema, la cifra digitale del messaggio è grande circa il doppio del messaggio cifrato.\\

La chiave segreta sarà $a$, con $0<a<p-1$, Alice calcola $\beta=g^a$ mod $p$.\\
La chiave pubblica è la tripla $(p, g, \beta)$ (gruppo ciclico, generatore e versione nascosta della chiave segreta).\\
Per firmare $m$ sceglie una $k \in \mathbb{Z}_{p-1}^*$, calcola $\gamma=g^k$ mod $p$ e $\delta = (m-a\gamma)k^{-1}$ mod $p-1$.\\
La coppia $(\gamma, \delta)$ è la firma di $m$.\\

Per ottenere $m$, mi basta fare le sostituzioni:
\begin{equation*}
	m = a\gamma + k\delta \text{ mod } p-1
\end{equation*}

Bob accetta la firma se e solo se:
\begin{equation*}
	\beta^\gamma \gamma^\delta \equiv g^m \text{ mod } p
\end{equation*}

Infatti se la firma è stata effettuata correttamente allora
\begin{equation*}
	\beta^\gamma \gamma^\delta \equiv g^{a \gamma} \cdot g^{k\delta} \equiv g^m \text{ mod } p
\end{equation*}

Mettiamo che Eve voglia produrre una firma per un messaggio $m$ senza conoscere il valore di $a$. Se Eve sceglie $\gamma$ e vuole calcolare il valore di $\delta$ in modo tale da rendere vera $\beta^\gamma \gamma^\delta \equiv g^m \text{ mod } p$. Il problema è che per farlo deve calcolarsi il seguente logaritmo discreto:
\begin{equation*}
	log_{\gamma}(g^m \beta^{-\gamma})
\end{equation*}

Invece se sceglie $\delta$ e vuole computare $\gamma$, deve risolvere l'equazione $\beta^{\gamma} \cdot \gamma^{\delta} \equiv ^m$ mod $p$ rispetto a $\gamma$ ma è infattibile anche questo problema.

Infine se sceglie sia $\gamma$ che $\delta$ e vuole calcolarsi un valore per $m$, deve calcolare il logaritmo discreto $\log_g\beta^{\gamma} \cdot \gamma^\delta$.

Il problema è che Eve può calcolare $\gamma, \delta$ e $m$ contemporaneamente!

(procedimento sbatti)

Però è possibile calcolare un messaggio $m$ e verificare che $(\gamma, \delta)$ sia una firma valida.
Quindi questo schema non è completamente rotto nel senso di aver trovato la chiave segreta, però così Eve può calcolare messaggi e firme.

Il problema è che la firma diventa troppo grande e in più El Gamal ha questa falla.

Sarebbe bello se avessimo una funzione che prende in input un messaggio di lunghezza arbitraria e ritorni un piccolo output in one-way.
Soluzione: funzioni Hash!
A questo punto anzichè firmare il messaggio firmo l'impronta hash del messaggio e non ho più il problema della dimensione visto che viene il doppio non più del messaggio.\\

Questo previene anche l'attacco al El Gamal visto che posso calcolare i valori di $\gamma, \delta$ e $m$, ma non so qual è il messaggio che produce quell'impronta ($m$)!

\subsection*{Funzioni di Hash}
\addcontentsline{toc}{subsection}{Funzioni di Hash}
Esse calcolano un'impronta (o message digest) per un dato in input.
Questa è piccola e di solito ha una lunghezza fissa (128 bit per MD5, 160 bit per SHA-1, 256 bit per SHA-256).
Possono essere usate per verificare l'integrità di un messaggio trasmesso.

Definizione: una famiglia di funzioni di hash è una quadrupla $(X,Y,K,H)$ dove:
\begin{itemize}
	\item $X$ è l'insieme dei possibili messaggi
	\item $Y$ è l'insieme delle possibili impronte
	\item $K$ è l'insieme delle possibili chiavi (noi non usiamo)
	\item $H=\{h_k:X \rightarrow Y | k \in K\}$ è l'insieme delle funzioni hash
\end{itemize}

L'insieme $X$ dei messaggi può essere finito o infinito (molto molto grande), mentre l'insieme delle impronte $Y$ è sempre finito.
Possiamo concludere che $|X| \geq |Y|$, anzi si assume che $|X| \geq 2|Y|$. In pratica è una funzione di compressione (da non confondere con archivi zip ecc).

I tre problemi difficili da risolvere quando si parla di funzioni di hash:

\begin{enumerate}
	\item \textbf{Preimage:}\\
	Input: funzione di hash $h : X \rightarrow Y$ e $y \in Y$\\
	Output $x \in X$ tale che $h(x) = y$\\
	
	In pratica ci viene chiesto di invertire la funzione di hash, $h$ dev'essere quindi one-way (data l'impronta dev'essere difficile calcolare un input che la generi)
	
	\item \textbf{Second preimage:}\\
	Input: funzione di hash $h : X\rightarrow Y$ e $x \in X$\\
	Output:  $x' \in X$ tale che $x \neq x'$ e $h(x') = h(x)$\\
	
	In pratica dato un messaggio $x$, e la sua impronta calcolata tramite una funzione $h$ dev'essere difficile trovare un messaggio $x'$ che abbia lo stesso valore in output della funzione
	
	\item \textbf{Collision:}\\
	Input: funzione di hash $h: X \rightarrow Y$\\
	Output: $x, x' \in X$ tali che $x' \neq x$ e che $h(x') = h(x)$\\
	
	Dev'essere difficile trovare due $x$ che abbiano la stessa impronta (esempio ti mando un file $x$ e poi ti dico guarda che in realtà ti ho mandato $x'$).\\
	Esso è più semplice da attaccare rispetto a second preimage, posso sempre provare ad attaccarlo.
\end{enumerate}

\subsubsection*{Algoritmo FindCollision}

Osservazione: collision è più facile da attaccare rispetto a second preimage. L'algoritmo assume che possiamo valutare la funzione $h$ per $q$ volte

\medskip
\begin{algorithmic}
	\State {Scegli a caso $X_0 \subseteq X$, con $|X_0| = q$}
	\ForAll{$x \in X_0$}
	\State {Calcola $y_x = h(x)$}
	\EndFor
	\If {$y_x = y_{x'}$ per qualche $x \neq x'$}
	\Return {$(x,x')$}
	\Else {}
	\Return {"Fail"}
	\EndIf
\end{algorithmic}
\medskip

Da notare che l'algoritmo non è completamente deterministico (a causa della scelta casuale su $X_0$, se sono fortunato mi ritorna una coppia altrimenti un errore).

