\documentclass[]{article}

\usepackage{graphicx}

\usepackage{amsmath}
\usepackage{amsfonts}

\usepackage{algorithm}

\usepackage[noend]{algpseudocode}

\usepackage[makeroom]{cancel}

\usepackage{lmodern}

\usepackage{empheq}

\newtheorem{theorem}{Teorema}

\usepackage{caption}

\usepackage{mathrsfs}

\usepackage{hyperref}

\usepackage{mathtools}

\usepackage{siunitx}

\usepackage{kbordermatrix}


\usepackage{upquote}

\renewcommand{\kbldelim}{(}% Left delimiter
\renewcommand{\kbrdelim}{)}% Right delimiter

\newcommand{\highlight}[1]{%
	\colorbox{yellow!20}{$\displaystyle#1$}}

\usepackage[most]{tcolorbox}

\usepackage{pgfplots}
\pgfplotsset{compat = newest}

\newtcbox{\mymath}[1][]{%
	nobeforeafter, math upper, tcbox raise base,
	enhanced, colframe=blue!30!black,
	colback=blue!30, boxrule=1pt,
	#1}

\newtheorem{dimostrazione}{Dimostrazione}

\makeatletter
\def\BState{\State\hskip-\ALG@thistlm}
\makeatother

\usepackage{tikz}
\tikzset{
	treenode/.style = {shape=rectangle, rounded corners,
		draw, align=center},
	root/.style     = {treenode, font=\ttfamily\normalsize},
	env/.style      = {treenode, font=\ttfamily\normalsize},
	dummy/.style    = {circle,draw}
}

\usepackage{multicol}

\setlength{\columnseprule}{0.3pt}
\def\columnseprulecolor{\color{black}}

\newenvironment{Figure}
{\par\medskip\noindent\minipage{\linewidth}}
{\endminipage\par\medskip}

\title{Teoria dell'informazione e crittografia\\ \vspace{3mm} \large{Riassunto videolezioni}}
\author{}
\date{} 

\begin{document}
	

\maketitle

\tableofcontents

\newpage

\input{lezione2}
\newpage
\input{lezione3}
\newpage
\input{lezione4}
\newpage
\input{lezione5}
\newpage
\input{lezione6}
\newpage
\input{lezione7}
\newpage
\input{lezione8}
\newpage
\input{lezione9}
\newpage
\input{lezione10}
\newpage
\input{lezione11}
\newpage
\input{lezione12}
\newpage
\input{lezione13}
\newpage
\input{lezione14}
\newpage
\input{lezione14-2}
\newpage
\input{lezione15}
\newpage
\input{lezione16}
\newpage
\input{lezione17}
\newpage
\input{lezione18}
\newpage
\input{lezione19}
\newpage
\section*{Lezione 20}
\addcontentsline{toc}{section}{Lezione 20}

\subsection*{Esponenziazione modulare}
Riprendiamo con un esempio: il gruppo moltiplicativo sarà $\mathbb{Z}_5^* = 1,2,3,4$, che è contenuto in $\mathbb{Z}_5 = {0,1,2,3,4}$.

2 è un generatore di $\mathbb{Z}_5^*$, infatti:

\[
\mathbb{Z}_5^* = \{2^0, 2^1, 2^2, 2^3\} = \{1,2,4,3\}
\]

Quindi possiamo definire la seguente funzione di esponenziazione modulare:
\begin{equation*}
f(z) = 2^z \text{ mod } 5
\end{equation*}

Nessuno è riuscito a trovare un algoritmo in P per trovare $z$ a partire da $f(z)$.
Dato $x$, però, posso calcolare $g^x \text { mod } p$ in tempo polinomiale (rispetto all'input). Ha senso considerare l'esponente $x$ fra 0 e $p-2$ (minore di 0 va all'inverso, maggiori di $p-2$, riconsidero gli stessi elementi visto che c'è modulo).
La dimensione dell'input è il numero di bit che servono per rappresentare gli elementi di $\mathbb{Z}_p^*$, quindi circa $n=log_2p$.\\
Possiamo quindi scrivere:
\begin{equation*}
	x = \sum_{j=0}^{n-1}x_j2^j
\end{equation*}
Dove $(x_{n-1}, ..., x_1, x_0)$ è la rappresentazione binaria di $x$.

Uno pseudocodice potrebbe essere:

\medskip
	\begin{algorithmic}
	\State \texttt{result} = 1
	\While {$x > 0$}
	\State \texttt{result} = (\texttt{result} * $g$)\text { mod }$p$
	\State $x = x - 1$
	\EndWhile
	\State{\textbf{return} \texttt{result}}	
\end{algorithmic}

\medskip
Però questo algoritmo chiede un numero esponenziale di iterazioni in quanto $x \approx 2^n$.\\


Però vale che:
\begin{equation*}
	g^x = g^{\sum_{j=0}^{n-1}x_j2^j} = \prod_{j=0}^{n-1} g^{x_j2^j}= \prod_{j=0}^{n-1}(g^{2^j})^{x_j} \; \; \text{ mod } p
\end{equation*}
Osservando l'ultima parte della formula posso calcolarmi tutti i valori di $g^{2^j}$ per $j \in \{0,1,...n-1\}$ e moltiplico fra loro solo quello per cui vale $x_j = 1$. Questo può essere fatto in maniera polinomiale attraverso il metodo \textit{square and multiply}.

\medskip

	\begin{algorithmic}
	\State {\texttt{result} $=1$}
	\For{$j=n-1 \text { downto } 0$}
	\State {\texttt{result} = (\texttt{result} * \texttt{result) mod $p$}}
	\If{$x_j = 1$}
	\State {\texttt{result} = \texttt{result} $* g$ mod $p$}
	\EndIf
	\EndFor
	\State{\textbf{return} \texttt{result}}		
	\end{algorithmic}

\medskip

L'inverso dell'esponenziazione modulare è chiamato \textit{logaritmo discreto}.
Il problema è così definito:
\begin{itemize}
	\item Numero primo $p$
	\item Generatore di $\mathbb{Z}_p^*$ chiamato $g$
	\item $y \in \mathbb{Z}_p^*$
\end{itemize}
Si computa una $x \in \{0, 1, ..., p-2\}$ tale per cui $g^x \equiv y \text{ mod } p$.

La computazione del logaritmo discreto è un problema intrattabile in tempo polinomiale. Per avere un idea della difficoltà del problema, prova a calcolare a mano i logaritmi base 3 dentro a $\mathbb{Z}_{113}^*$.

Abbiamo quindi trovato una scatola dove nascondere un valore $x$ che non può più essere aperta; in altre parole abbiamo trovato una one-way function. Manca però una \textbf{trapdoor} che consenta a Bob di trovare una soluzione in modo facile.

Un altro esempio di funzione che sembra essere one-way è la moltiplicazione fra numeri naturali: dati due numeri primi $p$ e $q$, calcolare il prodotto $n = p \dot q$.
Il problema inverso è la fattorizzazione: dato un numero interno $n$ che è il prodotto di due numeri primi, trovarli (ogni numero ammette un unica rappresentazione in fattori primi).\\



Un esempio di fattorizzazione:\\



RSA-768=12301866845301177551304949583849627207728535695953347921973224521\\
517264005072636575187452021997864693899564749427740638459251925573263034\\
537315482685079170261221429134616704292143116022212404792747377940806653\\
51419597459856902143413 = \\
334780716989568987860441698482126908177047949837137685689124313889828837\\
93878002287614711652531743087737814467999489 × 3674604366679959042824463\\
37996279526322791581643430876426760322838157396665112792333734171433968\\
10270092798736308917


L'algoritmo "banale" è quello che prova a dividere $n$ dato in input per tutti gli interi da 2 a $\sqrt{n}$, ma che equivale a un tempo di:

\begin{equation*}
	O(\sqrt{n}) = O(\sqrt{2^m}) = O(2^{\frac{m}{2}})
\end{equation*}

che è esponenziale rispetto a $m$ (il numero di bit).

\subsection*{Algoritmo di Diffie-Hellman}
\addcontentsline{toc}{subsection}{Algoritmo di Diffie-Hellman}

Vogliono sfruttare queste "mancanze" matematiche in crittografia. Alice e Bob scelgono (in maniera pubblica) un numero primo $q$ e un generatore $g$ del gruppo ciclico $\mathbb{Z}_q^*$. Poi alice sceglie a caso un $x_A$, con $0<x_A<q-1$, e lo tiene segreto. Bob allo stesso modo sceglie un $0 < x_B < q-1$ e lo tiene segreto.

Ora:
\begin{enumerate}
	\item Alice calcola $g^{x_A}$ mod $q$ e lo manda a Bob. In altre parole nasconde $x_A$ dentro una "cassetta".
	\item Bob calcola $g^{x_B}$ mod $q$ e lo manda ad Alice.
	\item Alice calcola $(g^{x_B})^{x_A} = g^{x_A \cdot x_B} = k$
	\item Bob calcola $(g^{x_A})^{x_B} = g^{x_B \cdot x_A} = k$ 
	\item $k$ è il valore della chiave segreta
\end{enumerate}
Eve vede solamente $q, g, g^{x_A}, g^{x_B}$, ma per trovare $x_A$ o $x_B$ deve risolvere un logaritmo discreto! Anche calcolare $g^{x_A \cdot x_B}$ partendo da $g^{x_A}$ e $g^{x_B}$, ma anche questo sembra un problema intrattabile.

\subsubsection*{Esempio}

Alice e Bob si mettono d'accordo utilizzando $q=25307$ e $g=2$ (che è un generatore di $\mathbb{Z}_{25307}^*)$.
Alice sceglie $x_A = 3578$ e Bob $x_B = 19956$.
Alice computa $g^{x_A} = 2^{3578} = 6113$ mod 25307 e lo manda a Bob. Allo stesso modo Bob calcola $g^{x_B} = 2^{19956} = 7984$ mod 25307 e lo manda ad Alice.
Quando Bob riceve $g^{x_A} = 6113$ da Alice, egli calcola $k=(g^{x_A})^{x_B} = 6113^{19956} = 3694$ mod 25307, e come lui anche Alice calcola $(g^{x_B})^{x_A} = 7984^{3578} = 3694$ mod 25307, che è la stessa chiave $k$ calcolata da Bob.
Ora Alice e Bob possono usare $k$ come chiave segreta in un crittosistema simmetrico

\subsection*{Crittosistema El Gamal}
\addcontentsline{toc}{subsection}{Crittosistema El Gamal}

Ogni utente sceglie un numero primo $q$, e considera un generatore $g$ del gruppo ciclico $\mathbb{Z}_q^*$.
Come prima sceglie un numero $a$ tale che $0 < a < q-1$ e computa $g^a$ mod $q$. La chiave segreta sarà $a$, e verrà nascosta dentro $g^a$ mod $q$. La chiave pubblica è $k_p(q, g, g^a)$. Per ottenere la chiave segreta dalla chiave pubblica dovremmo computare $log_gg^a$ mod $q$.\\


Alice vuole mandare un messaggio $m$ a Bob (questa $m$ dev'essere un elemento di $\mathbb{Z}_q^*$ quindi dev'essere minore di $log_2q$ bit, altrimenti lo divide in blocchi).
Lei prende la chiave pubblica di Bob $k_{pB} = (q, g, g^a)$. Ora sceglie un valore $0 < \ell < q-1$ e calcola $\gamma = g^{\ell}$ mod $q$ e $\delta = m \cdot (g^a)^{\ell}$ mod $q$.
Poi manda a Bob il testo cifrato $c=(\gamma, \delta)$.
Il valore $\ell$ rende i testi cifrati non correlabili fra loro (anche se sono sempre uguali).
La decifratura funziona in questo modo:
\begin{equation*}
\delta = m \cdot (g^a)^{\ell} \text{ mod } q.
\end{equation*}
\begin{equation*}
m = g^{-a\ell} \cdot \delta
\end{equation*}
\begin{equation*}
	m = g^{-a\ell} \cdot \delta = \gamma^{-a} \cdot \delta = \gamma^{q-1-a} \cdot \delta
\end{equation*}

Eve, come prima, dovrebbe risolvere un logaritmo discreto per rompere il crittosistema.

Questi crittosistemi a chiave pubblica però come si può vedere sono molto più lenti di quelli simmetrici (centinaia di volte più lenti).
Quindi si creano dei crittosistemi ibridi che sfruttano le potenzialità di entrambi: si usa un crittosistema a chiave pubblica per decidere una chiave segreta (chiave di sessione), e poi si utilizza questa in un crittosistema simmetrico. In questo modo si utilizza il crittosistema lento solo per passarsi qualche centinaia di bit.

\subsection*{RSA}
\addcontentsline{toc}{subsection}{RSA}

Dati due numeri primi molto grandi $p$ e $q$ della stessa dimensione più o meno (almeno 1024 bit ciascuno), ma valori molto diversi.\\
Si pone $n=pq$ ($n$ chiamato \textit{modulo} di RSA).
Ora si crea la funzione toziente di Nepero $\phi(n) = \phi(pq) = \phi(p)\phi(q) = (p-1)(q-1)$.\\
$\phi(n)$ è il numero di interi compresi fra 1 e $n$ che sono \textit{coprimi} con $n$ ($\phi(n) = |\{x:1 \leq x \leq n \text{ and MCD}(x,n) = 1\}|$).\\
In altre parole $\phi(n)$ è il numero di elementi contenuti in $\mathbb{Z}_N^*$.\\


Si sceglie un $d$ a caso tale che $1 < d < \phi(n)$ e che MCD$(d, \phi(n)) = 1$, quindi $d$ è invertibile modulo $\phi(n)$ (quindi è invertibile in $\mathbb{Z}_{\phi(N)}$)

Usando l'algoritmo di Euclide esteso si computa $e$ tale per cui $e \cdot d \equiv 1 \text{ mod } \phi(n)$ (che vuol dire $e \equiv d^{-1} \text{ mod } \phi(n)$)

La chiave pubblica è data dalla coppia $(n, e)$, mentre la chiave segreta sarà $d$. Ovviamente i valori di $p, q, $ e $\phi(n)$ devono rimanere segreti.

\begin{itemize}
	\item \textbf{Cifratura}:\\
	$c=m^e$ mod $n$\\
	Dove $m$ ed $e$ fanno parte della chiave pubblica. Se la lunghezza del messaggio $m$ da cifrare ha una dimensione più grande di $n$ dev'essere diviso in blocchi.
	\item \textbf{Decifratura}:\\
	$c^d$ mod $n = m$\\
	Essa funziona in questo modo:
	\begin{equation*}
		c^d \text{ mod } n \equiv (m^e)^d \equiv m^{ed} \text { mod } n
	\end{equation*}
	Ora, siccome $ed\equiv 1$ mod $\phi(n)$, per definizione di congruenza esiste una $k$ intera tale per cui $ed = 1 + k\phi(n)$, da cui:
	\begin{equation*}
		c^d \text{ mod } n \equiv m^{1+k\phi(n)} \equiv m \cdot (m^{\phi(n)})^k \text { mod } n
	\end{equation*}
Per il piccolo teorema di Fermat sappiamo che $m^{\phi(n)} \equiv 1$ mod $n$, quindi:
\begin{equation*}
	c^d \text{ mod } n \equiv m \cdot (1)^k \equiv m \cdot 1 \equiv m \text{ mod } n
\end{equation*}
	
\end{itemize}
Si sfrutta quindi l'esponenziazione modulare che è una permutazione one-way (la trapdoor è la chiave segreta).

\subsubsection*{Esempio}
Bob sceglie $p=101$ e $q=113$, da cui $n_B=pq=11413$ e $\phi(n_B) = (p-1)(q-1) = 11200$.\\
Ora deve scelgliere una $d_b$ fra 2 e $\phi(n_B) - 1 =11199$ tale che l'MCD fra $d_B$ e $\phi(n_B)$ sia uguale a 1.\\
Mettiamo che scelga $d_B=6597$.\\
Usando l'algoritmo di Euclude esteso, Bob si calcola
\begin{equation*}
	e_B \equiv d_B^{-1} \text{ mod } \phi(n_B) = 3533 \text { mod } 11200
\end{equation*}
Quindi la chiave pubblica è data dalla coppia $(n_B, e_B) = (11413, 3533)$,\\
la chiave privata è $d_B=6597$.\\

Se Alice volesse mandare il messaggio $m=9726$ e Bob, allora si calcola $c=9726^{3533}$ mod 11414 = 5761 e manda $c$ a Bob.\\
Egli recupera $m$ calcolando:
\begin{equation*}
	c^{d_B} = 5761^{6597} \text{ mod } 11413 = 9726
\end{equation*}

Ovviamente nella realtà le dimensioni dei numeri sono di almeno 1024 bit.\\


Quanto è difficile per Eve conoscere $\phi(n)$? \'E facile se conosciamo la fattorizzazione di $n$:
\begin{equation*}
	\phi(n) = n \prod_{p|n}(1 - \frac1p)
\end{equation*}
\'E difficile se partiamo da $n$ senza conoscere la sua fattorizzazione: in altre parole rompere RSA è equivalente a fattorizzare $n$.

Se Eve riuscisse a scoprire $\phi(n)$ può fattorizzare $n$:
\begin{itemize}
	\item Sa che $n=p\cdot q$ e che $\phi(n) = (p-1)(q-1) = pq - (p+q)+1=n-(p+q)+1$
	\item Calcola $p+q=n-\phi(n) + 1$
	\item Ora che conosce la somma e il prodotto di $p$ e di $q$, li calcola risolvendo la seguente equazione quadratica $x^2 -(p+q) = 0$.
\end{itemize}

I valori di $p$ e $q$ non dovrebbero essere troppo vicini fra loro (altrimenti si avvicinano alla $\sqrt{n}$ e quindi provando a cercare intorno a quel valore troviamo uno dei due fattori e fattorizziamo $n$).
Vediamo un caso in cui $p$ e $q$ sono vicini:
\begin{itemize}
	\item Assumiamo che $p>q$
	\item Se i due sono vicini, allora $\frac{p-q}{2}$ è piccolo, e $\frac{p+q}{2}$ è leggermente più grande di $\sqrt{n}$.
	\item Dalla sequenze uguaglianza (che è valida in generale):
	\begin{equation*}
		\frac{(p+q)^2}{4} - n = \frac{(p-q)^2}{4}
	\end{equation*}
	Deduciamo che $\frac{(p+q)^2}{4}$ è un quadrato perfetto.
	\item Allora cerchiamo degli interi $x > \sqrt{n}$ tali per cui $x^2-n$ è un quadrato perfetto che chiamiamo $y^2$.
	\item Da $y^2 = x^2 -n$ otteniamo:
	\begin{equation*}
		n = x^2-y^2 =(x+y)(x-y)
	\end{equation*}
	E quindi abbiamo fattorizzato! $p=x+y$ e $q=x-y$
\end{itemize}


Dovremmo anche occuparci del valore di $\phi{n}$: assumiamo che il MCD è grande, allora
\begin{equation*}
	u = \text{mcm}(p-1,q-1) = \frac{(p-1)(q-1)}{\text{MCD}(p-1,q-1)} = \frac{\phi(n)}{\text{mcm(p-1, q-1)}}
\end{equation*}
Dalle proprietà dei campi se $d' \equiv e^{-1} \text{ mod } u $ si può utilizzare $d'$ per decifrare al posto di $d$. Ma siccome $u$ è piccolo, si può trovare a tentativi. Quindi è meglio se $p-1$ e $q-1$ abbiano divisori grandi.\\

Un altro errore è utilizzare lo stesso modulo $n$ per un gruppo di utenti.
Alice manda
\begin{equation*}
	c_1 = m^{e_1} \text { mod } n
\end{equation*}
\begin{equation*}
	c_2 = m^{e_2} \text { mod } n
\end{equation*}
Se MCD$(e_1, e_2) = 1$ (sono coprimi fra loro) allora Eve può usare il teorema di Euclide esteso e calcola $r$ e $s$ tali per cui $re_1 + se_2 = 1$.
Una volta calcolato $r$ e $s$ Eve può calcolare:
\begin{equation*}
	c_1^rc_2^s \equiv m^{re_1}m^{se_2} \equiv m^{re_1 + se_2} \equiv m \text { mod } n
\end{equation*}

Anche utilizzare esponenti piccolo per valori diversi del modulo può essere un problema: mettiamo che un utente vuole mandare $m$ a tre utenti $A,B,C$ usando tre moduli diversi:
\begin{equation*}
	c_A = m^3 \text{ mod } n_A
\end{equation*}
\begin{equation*}
	c_B = m^3 \text{ mod } n_B
\end{equation*}
\begin{equation*}
	c_C = m^3 \text{ mod } n_C
\end{equation*}

Se $n_A,n_B$ e $n_C$ sono coprimi allora Eve può usare il teorema dei Cinesi per calcolare un valore $m$ la cui radice cubica intera ritorna $m$.\\

Rompere RSA equivale a fattorizzare $n$.

\subsubsection*{Randomized RSA}

Un modo robusto di utilizzare RSA è la seguente: supponiamo di avere una sequenza di singoli bit da cifrare. Allora prendiamo la chiave pubblica di RSA $(n,e)$, allora se cifriamo 0 rimane 0, e così anche 1:
\begin{equation*}
	0^e \equiv 0 \text { mod } n
\end{equation*}
\begin{equation*}
	1^e \equiv 1 \text { mod } n
\end{equation*}
quindi il testo cifrato è uguale al testo in chiaro!
Si potrebbe risolvere questa cosa creando pacchetti di messaggi, ma magari non è possibile perchè devo mandarli immediatamente. Qualcuno ha però dimostrato che il bit meno significativo del testo in chiaro è un hard-core bit per l'esponenziazione modulare $m^e$ mod $n$ (quindi è difficile calcolare il bit meno significativo del testo in chiaro senza trapdoor, ovvero $d$ aka one-way function)
L'idea è quella di mettere il bit da cifrare come bit meno significativo, e gli altri bit vengono messi a random!
\subsubsection*{Esempio}
Alice vuole mandare a Bob un bit: prende la chiave pubblica di Bob $(n_B, e_B)$.
Sceglie poi un intero random $x < \frac{n_B}{2}$ (quindi $2x < n_B$).
Poi manda a Bob $y=(2x+b)^{e_B}$ mod $n_B$.\\
Bob quando riceve $y$ computa $y^{d_B}$ mod $n_B = 2x + b$ e prende il bit meno significativo del risultato.\\

Ovviamente questo è fattibile per cifrare qualche bit ogni tanto, e non per cifrare un bit alla volta un film.


\newpage
\section*{Lezione 21}
\addcontentsline{toc}{section}{Lezione 21}

\subsection*{Firme digitali e funzioni di hash}
\addcontentsline{toc}{subsection}{Firme digitali e funzioni di hash}

Una firma digitale è una quintupla $(P,A,K,S,V)$ dove:
\begin{itemize}
	\item $P$ è l'insieme di tutti i possibili messaggi che vorremmo firmare
	\item $A$ è l'insieme di tutte le possibili firme
	\item $K$ è l'insieme di tutte le possibili chiavi
	\item $S=\{\text{sig}:P\times K \rightarrow A\}$ è l'insieme delle funzioni di firma
	\item $V=\{\text{ver}:P\times A \times K \rightarrow \{T, F\}\}$ è l'insieme delle funzioni di verifica
\end{itemize}

Come nelle funzioni di cifratura e decifratura, scegliendo una chiave $k \in K$ essa diventa un parametro:
\begin{equation*}
	\text{sig}_k:P \rightarrow A
\end{equation*}
\begin{equation*}
	\text{ver}_k:P \times A \rightarrow \{T, F\}
\end{equation*}

La funzione di verifica mi dice sempre T o F.

La coppia $(x,y)$ è chiamata un messaggio firmato.

Diversamente da un autografo (che è fatto a mano), le firme:
\begin{itemize}
	\item la firma $y$ è separata dal documento $x$
	\item la firma non è sempre la stessa: dipende dal documento, quindi ce ne sono diverse per diversi documenti
\end{itemize}

La firma digitale serve per autenticare l'originatore del messaggio: solo chi conosce una certa informazione segreta può produrre la firma del messaggio.

Per firmare serve una chiave segreta, per verificare si usa una chiave pubblica; quindi il meccanismo è simile a quello dei crittosistemi a chiave pubblica ma al contrario.

Attenzione: autentica il generatore del messaggio ma non chi lo ha spedito!

\subsubsection*{Esempio}
Alice vuole firmare un messaggio $m$:
\begin{itemize}
	\item Lei scegli una coppia $(sig_k, ver_k)$ di algoritmi
	\item Mantiene segreta $sig_k$ e rende $ver_k$ pubblica
	\item Calcola $\sigma=sig_k(m)$
\end{itemize}
A questo punto Bob vuole verificare la firma prodotta da Alice:
\begin{itemize}
	\item Considera la coppia $(m, \sigma)$
	\item Prende l'algoritmo $ver_k$ e accetta la valida se e solo se $ver_k(m, \sigma) = T$.
\end{itemize}

Eve vorrebbe:
\begin{itemize}
	\item Rompere totalmente il sistema: in qualche modo riesce a determinare la chiave segreta $k$ e quindi firmare documenti al suo posto
	\item Existential forgery: dopo aver osservato qualche coppia $(x_1, y_1), ..., (x_i, y_i)$ di messaggi con le rispettive firme, quando un nuovo messaggio arriva Eve può produrre una valida firma (valida solo per quel messaggio o per una classe di messaggi che hanno qualche similarità).
\end{itemize}

Quando Alice vuole mandare un messaggio firmato $m$ (non cifrato, solo firmato) a Bob, lei ha una funzione di cifratura $E_A$ e una funzione di decifratura $D_A$. Mantiene segreta $D_A$ mentre $E_A$ è pubblica.\\
A questo punto se vuole firmare $m$ usa $D_A(m)$ e la manda a Bob. Egli per verificare la firma prende l'algoritmo pubblico $E_A$ e controlla che \begin{equation*}
	E_A(D_A(m)) = m
\end{equation*}
E accetta la firma se e solo se la funzione di verifica ritorna vero.

Di solito però si firma un documento cifrato, quindi Alice calcola $D_A(E_B(m))$ utilizzando $E_B$ come algoritmo di cifratura pubblico di Bob. \\
Bob infine per trovare $m$ calcolerà:
\begin{equation*}
	D_B(E_A(D_A(E_B(m)))) = m
\end{equation*}
Quindi controlla la firma e poi decifra il messaggio.\\


Così però non va bene: se Eve intercettasse potrebbe prendere la firma e impersonificare Alice mandando $D_E(E_B(m))$. Quindi è meglio cifrare anche la firma, così solo Bob può leggere la firma.

\subsection*{Schema El Gamal}
\addcontentsline{toc}{subsection}{Schema El Gamal}

La base su cui è stato prodotto DSA (Digital Signature Algorithm) che è algoritmo usato ora.\\
Lo schema di El-Gamal non è deterministico: ogni messaggio (anche uguale) ha sempre una firma diversa. Come il suo crittosistema, la sicurezza si basa sui logaritmi discreti.
\begin{itemize}
	\item Si parte da un numero primo $p$
	\item Si trova un generatore $g$ del generatore del gruppo ciclico $\mathbb{Z}_p^*$
	\item I messaggi da cifrare sono elementi di $\mathbb{Z}_p^*$
	\item Le firme sono coppie $(\gamma, \delta)$ con $\gamma \in \mathbb{Z}_p^*$ e $\delta \in \mathbb{Z}_{p-1}$ dove $\mathbb{Z}_{p-1}$ non sarà un campo ($p-1$ è pari) ma un anello o boh 
\end{itemize}

Il problema, come nel crittosistema, la cifra digitale del messaggio è grande circa il doppio del messaggio cifrato.\\

La chiave segreta sarà $a$, con $0<a<p-1$, Alice calcola $\beta=g^a$ mod $p$.\\
La chiave pubblica è la tripla $(p, g, \beta)$ (gruppo ciclico, generatore e versione nascosta della chiave segreta).\\
Per firmare $m$ sceglie una $k \in \mathbb{Z}_{p-1}^*$, calcola $\gamma=g^k$ mod $p$ e $\delta = (m-a\gamma)k^{-1}$ mod $p-1$.\\
La coppia $(\gamma, \delta)$ è la firma di $m$.\\

Per ottenere $m$, mi basta fare le sostituzioni:
\begin{equation*}
	m = a\gamma + k\delta \text{ mod } p-1
\end{equation*}

Bob accetta la firma se e solo se:
\begin{equation*}
	\beta^\gamma \gamma^\delta \equiv g^m \text{ mod } p
\end{equation*}

Infatti se la firma è stata effettuata correttamente allora
\begin{equation*}
	\beta^\gamma \gamma^\delta \equiv g^{a \gamma} \cdot g^{k\delta} \equiv g^m \text{ mod } p
\end{equation*}

Mettiamo che Eve voglia produrre una firma per un messaggio $m$ senza conoscere il valore di $a$. Se Eve sceglie $\gamma$ e vuole calcolare il valore di $\delta$ in modo tale da rendere vera $\beta^\gamma \gamma^\delta \equiv g^m \text{ mod } p$. Il problema è che per farlo deve calcolarsi il seguente logaritmo discreto:
\begin{equation*}
	log_{\gamma}(g^m \beta^{-\gamma})
\end{equation*}

Invece se sceglie $\delta$ e vuole computare $\gamma$, deve risolvere l'equazione $\beta^{\gamma} \cdot \gamma^{\delta} \equiv ^m$ mod $p$ rispetto a $\gamma$ ma è infattibile anche questo problema.

Infine se sceglie sia $\gamma$ che $\delta$ e vuole calcolarsi un valore per $m$, deve calcolare il logaritmo discreto $\log_g\beta^{\gamma} \cdot \gamma^\delta$.

Il problema è che Eve può calcolare $\gamma, \delta$ e $m$ contemporaneamente!

(procedimento sbatti)

Però è possibile calcolare un messaggio $m$ e verificare che $(\gamma, \delta)$ sia una firma valida.
Quindi questo schema non è completamente rotto nel senso di aver trovato la chiave segreta, però così Eve può calcolare messaggi e firme.

Il problema è che la firma diventa troppo grande e in più El Gamal ha questa falla.

Sarebbe bello se avessimo una funzione che prende in input un messaggio di lunghezza arbitraria e ritorni un piccolo output in one-way.
Soluzione: funzioni Hash!
A questo punto anzichè firmare il messaggio firmo l'impronta hash del messaggio e non ho più il problema della dimensione visto che viene il doppio non più del messaggio.\\

Questo previene anche l'attacco al El Gamal visto che posso calcolare i valori di $\gamma, \delta$ e $m$, ma non so qual è il messaggio che produce quell'impronta ($m$)!

\subsection*{Funzioni di Hash}
\addcontentsline{toc}{subsection}{Funzioni di Hash}
Esse calcolano un'impronta (o message digest) per un dato in input.
Questa è piccola e di solito ha una lunghezza fissa (128 bit per MD5, 160 bit per SHA-1, 256 bit per SHA-256).
Possono essere usate per verificare l'integrità di un messaggio trasmesso.

Definizione: una famiglia di funzioni di hash è una quadrupla $(X,Y,K,H)$ dove:
\begin{itemize}
	\item $X$ è l'insieme dei possibili messaggi
	\item $Y$ è l'insieme delle possibili impronte
	\item $K$ è l'insieme delle possibili chiavi (noi non usiamo)
	\item $H=\{h_k:X \rightarrow Y | k \in K\}$ è l'insieme delle funzioni hash
\end{itemize}

L'insieme $X$ dei messaggi può essere finito o infinito (molto molto grande), mentre l'insieme delle impronte $Y$ è sempre finito.
Possiamo concludere che $|X| \geq |Y|$, anzi si assume che $|X| \geq 2|Y|$. In pratica è una funzione di compressione (da non confondere con archivi zip ecc).

I tre problemi difficili da risolvere quando si parla di funzioni di hash:

\begin{enumerate}
	\item \textbf{Preimage:}\\
	Input: funzione di hash $h : X \rightarrow Y$ e $y \in Y$\\
	Output $x \in X$ tale che $h(x) = y$\\
	
	In pratica ci viene chiesto di invertire la funzione di hash, $h$ dev'essere quindi one-way (data l'impronta dev'essere difficile calcolare un input che la generi)
	
	\item \textbf{Second preimage:}\\
	Input: funzione di hash $h : X\rightarrow Y$ e $x \in X$\\
	Output:  $x' \in X$ tale che $x \neq x'$ e $h(x') = h(x)$\\
	
	In pratica dato un messaggio $x$, e la sua impronta calcolata tramite una funzione $h$ dev'essere difficile trovare un messaggio $x'$ che abbia lo stesso valore in output della funzione
	
	\item \textbf{Collision:}\\
	Input: funzione di hash $h: X \rightarrow Y$\\
	Output: $x, x' \in X$ tali che $x' \neq x$ e che $h(x') = h(x)$\\
	
	Dev'essere difficile trovare due $x$ che abbiano la stessa impronta (esempio ti mando un file $x$ e poi ti dico guarda che in realtà ti ho mandato $x'$).\\
	Esso è più semplice da attaccare rispetto a second preimage, posso sempre provare ad attaccarlo.
\end{enumerate}

\subsubsection*{Algoritmo FindCollision}

Osservazione: collision è più facile da attaccare rispetto a second preimage. L'algoritmo assume che possiamo valutare la funzione $h$ per $q$ volte

\medskip
\begin{algorithmic}
	\State {Scegli a caso $X_0 \subseteq X$, con $|X_0| = q$}
	\ForAll{$x \in X_0$}
	\State {Calcola $y_x = h(x)$}
	\EndFor
	\If {$y_x = y_{x'}$ per qualche $x \neq x'$}
	\Return {$(x,x')$}
	\Else {}
	\Return {"Fail"}
	\EndIf
\end{algorithmic}
\medskip

Da notare che l'algoritmo non è completamente deterministico (a causa della scelta casuale su $X_0$, se sono fortunato mi ritorna una coppia altrimenti un errore).


\newpage
\section*{Lezione 22}
\addcontentsline{toc}{section}{Lezione 22}

Vogliamo capire qual è la probabilità di successo di FindCollision, proviamo a riscriverlo in maniera più facile da analizzare:

\medskip
\begin{algorithmic}
	\State {Scegli a caso $X_0 \subseteq X$, con $|X_0| = q$}
	\State {fingerprints = []}
	\ForAll{$x \in X_0$}
	\State {Calcola $y_x = h(x)$}
	\EndFor
	\If {$y_x \in$ fingerprints}
	\Return {$(x,x')$}
	\Else { add $y_x$ to fingerprints}
	\EndIf
	\Return {"Fail"}
\end{algorithmic}

Teorema: diciamo che $M = |Y|$ (insieme delle possibili impronte). La probabilità di successo $\epsilon$ di FindCollision è 
\begin{equation*}
	\epsilon = 1 - (\frac{M-1}{M})(\frac{M-2}{M})...(\frac{M-q}{M})
\end{equation*}

Dimostrazione: dato $X_0 = \{x_1, ..., x_q\}$, per ogni $i$ compresa in $0 < i \geq q$ si definisce $E_i$ l'evento $h(x_i) \notin \{h(x_1), ..., h(x_{i-1})\}$, allora Prob[$E_1$] = 1, e le altre:
\begin{equation*}
	\text{Prob}[E_1 \land E_2 \land ... \land E_q] = (\frac{M-1}{M})(\frac{M-2}{M})...(\frac{M-q+q}{M})
\end{equation*}
Questa è la probabilità che fallisca, quindi la $\epsilon$ sopra è quella che abbia succcesso.

\subsubsection*{Il paradosso del compleanno}

Riscrivo la probabilità di non avere collisioni come:

\begin{equation*}
	(1 - \frac1M)(1 - \frac2M) ... (1 - \frac{q-1}{M}) = \prod_{i=1}^{q-1}(1 - \frac{i}{M})
\end{equation*}

Ora approssimiamo $1-x \approx e^{-x}$ , quindi
\begin{equation*}
	\text{Prob[NoCollisions]} \approx e^{-\frac{i}{M}} = e^{-\frac1M \sum_{i=1}{q-1}i} = e^{-\frac{q(q-1)}{2M}}
\end{equation*}
quindi la probabilità di trovare (almeno) una collisione è
\begin{equation*}
	\text{Prob[AtLeastOneCollision]} \approx 1 - e^{-\frac{q(q-1)}{2M}} = \epsilon
\end{equation*}


Questa relazione lega: probabilità di avere una collisioni ($\epsilon$), numero di possibili impronte($M$) e lo "sforzo" computazionale (numero di query) che siamo disposti a fare ($q$) nel cercare una collisione.\\

Ora proviamo ad estrarre lo sforzo computazionale:

\begin{equation*}
	e^{-\frac{q(q-1)}{2M}} = 1 - \epsilon
\end{equation*}

\begin{equation*}
	-\frac{q(q-1)}{2M} = ln(1-e)
\end{equation*}

\begin{equation*}
	q^2 - q = 2M ln(\frac{1}{1-\epsilon})
\end{equation*}

\begin{equation*}
	q \approx \sqrt{2Mln(\frac{1}{1-\epsilon})}
\end{equation*}

Quindi questo valore rappresenta (circa) il numero di query che devo fare per trovare almeno una collisione con probabilità $\epsilon$ che ha $M$ possibili impronte.

Proviamo $\epsilon= \frac12$:
\begin{equation*}
	q \approx \sqrt{2Mln2}
\end{equation*}
\begin{equation*}
	q \approx 1.17\sqrt{M}
\end{equation*}

Quindi le hash di circa $\sqrt{M}$ elementi abbiamo una collisione con probabilità $\frac12$.\\


Con 40 bit di impronta abbiamo $M=2^{40} \rightarrow \sqrt{M} = 2^{20} \approx 10^6$ (Non sicuro! 1 milione li provo in tra).


Con 128 bit di impronta abbiamo $M=2^{128} \rightarrow \sqrt{M} = 2^{64} \approx 10^6$ (Sicuro per poco tempo).

Di solito si utilizzano \textit{almeno} 160 bit.\\

Questa è la stessa cosa di chiedersi quante persone servono per avere la probabilità $\frac12$ che due di loro siano nate nello stesso giorno:

\begin{equation*}
\sqrt{365} \cdot 1.17 \approx 23 \text{ lol}
\end{equation*}

\subsubsection*{Costruzione di funzioni di hash iterate}

Data una funzione di compressione $\{0,1\}^{m+t} \rightarrow \{0,1\}^m$ con $t \geq 1$

Le tecnica è composta da tre fasi:
\begin{itemize}
	\item \textbf{Pre-computazione}\\
	Data una stringa in input $x$ che ha lunghezza almeno $m+t+1$, creiamo una stringa $y$ tale per cui $|y| \equiv 0 $ mod $t$ (quindi che abbia lunghezza multiplo di $t$).\\
	Di solito si usa una funzione di padding $pad(x)$ che aggiunge bit alla fine uguali a zero, poi dividiamo $y$ in $r$ blocchi, ognuno di $t$ bit:
	\begin{equation*}
		y = y_1 || y_2 || ... || y_r
	\end{equation*}
Questa fase di pre-computazione dovrebbe assicurare che la funzione $f(x) = y$ computata sia la più vicina possibile ad una iniettiva (input diversi hanno output diversi) se no è più facile trovare collisioni. 
Questa condizione non è difficile da soddifare in quanto $|y| = rt \geq |x|$.

\item \textbf{Computazione}:\\
Definiamo un vettore di inizializzazione $z_0 = IV$ (IV = Initialization Vector) che di solito è noto.
\begin{equation*}
	z_0 = IV
\end{equation*}
\begin{equation*}
	z_1 = compress(z_0 || y_1)
\end{equation*}
\begin{equation*}
	z_2 = compress(z_1 || y_2)
\end{equation*}
\begin{center}
	...
\end{center}
\begin{equation*}
	z_r = compress(z_{r-1} || y_r)
\end{equation*}

\item \textbf{Trasformazione output}:\\
Si trasforma l'output (opzionale)
\end{itemize}

Questa costruzione generale è stata specializzata da Merkle e Damgard in due modi.\\
Con questa costruzione possiamo dimostrare che se la funzione \textit{comprimi} è resistente alle collisioni allora anche la funzione \textit{hash} iterata è resistente alle collisioni.\\


Si prende un input $x$ di lunghezza $n$, con $|x| = n \geq m + t +1$ per $t \geq 2$. Si divide $x$ in $k$ blocchi, ognuno di $t-1$ bits:
\begin{equation*}
	x = x_1 || x_2 || ... || x_k
\end{equation*}

con $k= [\frac{n}{t-1}]$ e $|x_k| = t - 1 - d$ con $0 \leq d \leq t -2$ ($d$ sarebbe la lunghezza del pezzo che manca per creare padding, si aggiungono tanti 0).

Si aggiunge un blocco in $k+1$ che contiene il valore del numero $d$ (così so quanti 0 ho aggiunto per padding).

mhhh

\subsubsection*{Alcune funzioni di Hash}

La prima funzione di hash è stata MD4 nel 1990, MD5 è una modifica di MD4 che è più resistente agli attacchi. Però queste versioni sono suscettibili all'attacco del compleanno.\\
Viene introdotto SHA come standard, che poi diventa SHA-1 dopo aver corretto una falla.
Anche in SHA-1 però è stata trovata una collisione, quindi ora bisogna usare SHA-256.\\

SHA-1 
non chiesto all'orale come funziona


\subsection*{Digital Signature Algorithm}
\addcontentsline{toc}{subsection}{Digital Signature Algorithm}

\'E stato preso El Gamal e sono state effettuate alcune modifiche.

"i dettagli non ve li chiedo all'orale"

=)

min2s



\end{document}
